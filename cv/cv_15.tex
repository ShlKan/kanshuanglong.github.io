%%%%%%%%%%%%%%%%%%%%%%%%%%%%%%%%%%%%%%%
% Wenneker Resume/CV
% LaTeX Template
% Version 1.1 (19/6/2016)
%
% This template has been downloaded from:
% http://www.LaTeXTemplates.com
%
% Original author:
% Frits Wenneker (http://www.howtotex.com) with extensive modifications by 
% Vel (vel@LaTeXTemplates.com)
%
% License:
% CC BY-NC-SA 3.0 (http://creativecommons.org/licenses/by-nc-sa/3.0/
%
%%%%%%%%%%%%%%%%%%%%%%%%%%%%%%%%%%%%%%

%----------------------------------------------------------------------------------------
%	PACKAGES AND OTHER DOCUMENT CONFIGURATIONS
%----------------------------------------------------------------------------------------

\documentclass[a4paper,10pt]{memoir} % Font and paper size

\usepackage{bibentry}
\makeatletter\let\saved@bibitem\@bibitem\makeatother
\usepackage{hyperref}
\makeatletter\let\@bibitem\saved@bibitem\makeatother



\newcommand\publication[1]{%
	\smallskip\par\hangpara{1.5em}{1}\bibentry{#1}\smallskip
}


\input{structure.tex} % Include the file specifying document layout and packages

%----------------------------------------------------------------------------------------
%	NAME AND CONTACT INFORMATION 
%----------------------------------------------------------------------------------------

\userinformation{ % Set the content that goes into the sidebar of each page
\begin{flushright}
% Comment out this figure block if you don't want a photo
\includegraphics[width=0.9\columnwidth]{shuanglong}\\[\baselineskip] % Your photo
\small % Smaller font size
Shuanglong Kan \\ % Your name
\href{kanshuanglong@outlook.com}{kanshuanglong@outlook.com} \\ % Your email address
+49 1738938642
    
\Sep % Some whitespace
\textbf{Address} \\
Nelkenstr. 114A\\ % Address 1
67691 Hochspeyer	 \\ % Address 2
Germany \\ % Address 3
\vfill % Whitespace under this block to push it up under the photo
\end{flushright}
}

%----------------------------------------------------------------------------------------

\begin{document}
	\nobibliography{cv.bib}
	\bibliographystyle{plain}
	

\userinformation % Print your information in the left column

\framebreak % End of the first column

%----------------------------------------------------------------------------------------
%	HEADING
%----------------------------------------------------------------------------------------

\cvheading{Shuanglong Kan} % Large heading - your name

\cvsubheading{Computer Scientist} % Subheading - your occupation/specialization

%----------------------------------------------------------------------------------------
%	Objective
%----------------------------------------------------------------------------------------

\aboutme{Background}{My research focusing on solving safety and security challenges in programming languages. Some of my representative works are as follows:
	\begin{itemize}
		\item A certified SMT solver (\href{https://dl.acm.org/doi/10.1145/3497775.3503691}{CertiStr}) for string data types. The solver algorithms are formalized and proved correct in \href{https://isabelle.in.tum.de/}{Isabelle/HOL}. Leveraging Isabelle/HOL to automatically generate the executable code in OCaml from the verified algorithms.
		\item A certified compiler for SIGNAL language. SIGNAL is a synchronous language designed for embedded systems. The low-level code generation procedure is mathematically verified by \href{http://www.event-b.org/}{Event-B}. Event-B is a formal modeling and verification tool supporting refinement-based system development.
		\item A front-end for C language to automatically support code instruments for testing. For some special testing requirements, the tool automatically inserts testing code.
		\item The verification-oriented compilers for Rust and Solidity languages. The compilers translate Rust or Solidity programs to mathematical formulas, and we then leverage mathematical tools to reason the  programs' correctness automatically.
	\end{itemize}
	Most of my works are published in top-tier or professional formal methods conferences, such as POPL (ACM SIGPLAN Symposium on Principles of Programming Languages), ISSTA (ACM SIGSOFT International Symposium on Software Testing and Analysis), S\&P (IEEE Symposium on Security and Privacy), and CPP (Certified Programs and Proofs). 
}

%----------------------------------------------------------------------------------------
%	EDUCATION
%----------------------------------------------------------------------------------------

\CVSection{Education}

%------------------------------------------------

\CVItem{Sep. 2011 - Jun. 2017, Nanjing University of Aeronautics and Astronautics, China
}{Doctor of Computer Science. Supervisor: Prof. Zhiqiu Huang} 

Thesis: Research on the Trustworthy Code Generation from SIGNAL

\begin{itemize}
	\item \href{http://polychrony.inria.fr/}{SIGNAL} is a mathematical modeling language for distributed reactive system.
	\item Design a multi-threaded code generation from SIGNAL.
	\item Mathematically prove  the correctness of the code generation in \href{http://www.event-b.org/}{Event-B}.
	\item Develop model checking techniques for state/event systems modeled by SIGNAL.
\end{itemize}
~\\
%------------------------------------------------
\CVItem{Jan.2016 - Jun. 2016, IRIT, University Toulouse III, France
}{Visiting Researcher. Supervisor: Prof. Jean-Paul Bodeveix and Mamoun Filali}

Project: Certified code generation from SIGNAL.
~\\



\CVItem{Sep. 2007 - Jun. 2011, Nanjing University of Aeronautics and Astronautics, China}{Bachelor of Computer Science}
%------------------------------------------------

\Sep % Extra whitespace after the end of a major section

%----------------------------------------------------------------------------------------
%EXPERIENCE
%----------------------------------------------------------------------------------------

\CVSection{Work Experience}

%------------------------------------------------

\CVItem{01.11.2022 - now, Freelancer,  one customer is \href{https://www.certik.com/}{Certik},  a Web3 company focusing smart contract security.}{
Certik is a leading web3 security company located in New York, US.\\
Project:  A verification-oriented compiler for Solidity language
\begin{itemize}
	\item Design the architecture of smart contract formal verification tools. 
	\item Design reusable Intermediate Representation (IR) for smart contracts formal verification.
	\item Develop model checking techniques for smart contracts.
	\item Develop deductive verification techniques for smart contracts.
\end{itemize}
Smart contracts are snippets of code integrated into blockchains to support financial activities. 
Vulnerabilities in smart contracts may yield billions of dollars being hacked.
}


\CVItem{15.10. 2019 - 31.10.2022, \textit{Research Assistant}, TU Kaiserslautern, Germany}
{
Project: A Certified mathematical reasoning tool for string types in programming languages. Supervisor Anthony W. Lin
\begin{itemize}
	
	\item I developed a certified string solver using Isabelle  proof assistant, which can be widely used in programming language verification. Amazon AWS and Microsoft Azure also use string solvers heavily for access control policies.

\item Certified Software means the software is mathematically proved by interactive proof assistant, such as \href{https://coq.inria.fr/}{Coq} and \href{https://isabelle.in.tum.de/}{Isabelle}. 
	

\end{itemize}

More info about
SMT solvers, such as \href{https://github.com/Z3Prover/z3}{z3}, and \href{https://cvc5.github.io/}{CVC5} are widely used in providing automated reasoning power for computer systems. Tools, such as \href{https://frama-c.com/}{Frama-C},  Event-B,  Software Model checkers (such as \href{https://monteverdi.informatik.uni-freiburg.de/tomcat/Website/}{Ultimate}, 
\href{https://cpachecker.sosy-lab.org/}{CPAChecker}, and \href{https://www.cprover.org/cbmc/}{CBMC}) are all based on SMT solvers. Our string solver can be integrated into the existing SMT solvers. 
}

\clearpage % Start a new page

\userinformation % Print your information in the left column

\framebreak % End of the first column

\CVItem{01.09.2017 - 30.09.2019, \textit{Research Fellow}, Nanyang Technological University, Singapore}{
	Project: A verification-oriented compiler for Rust language.
	Supervisor Yang Liu and Shang-wei Lin
	\begin{itemize}
		\item Rust is a programming language aiming for replacing C and non-memory errors. 
		\item I developed the security vulnerability detection tools for Rust  using \href{https://kframework.org/}{K-Framework}, which is a rewriting-logic based semantics modeling tool.
	\end{itemize}
}

%------------------------------------------------




%----------------------------------------------------------------------------------------
%	NEW PAGE DELIMITER
%	Place this block wherever you would like the content of your CV to go onto the next page
%----------------------------------------------------------------------------------------









%----------------------------------------------------------------------------------------

\Sep % Extra whitespace after the end of a major section

\CVSection{Fellowships/Awards}

%------------------------------------------------

\CVItem{2016, \textit{Visiting PhD student Scholarship}, Nanjing University of Aeronautics and Astronautics}{The scholarship covers all my expense for the academic visiting in Toulouse, France.}


\CVItem{2017, \textit{Excellent Doctor degree dissertation}, Nanjing University of Aeronautics and Astronautics}{For my PhD thesis: Research on the Trustworthy Code Generation from SIGNAL.}


\CVItem{2019, \textit{ISSTA Distinguished Paper Award}}{The ACM SIGSOFT International Symposium on Software Testing and Analysis (ISSTA) is the leading research symposium on software testing and analysis.}


\CVItem{2022, \textit{CPP Distinguished Paper Award}}{Certified Programs and Proofs (CPP) is an international conference on practical and theoretical topics in all areas that consider formal verification and certification as an essential paradigm for their work}

\CVItem{2023, \textit{SMT-COMP 2023 (Satisfiability Modulo Theories Competition) Winner of the Logic String}}{SMT-COMP is the most important contest in SMT community.}
%------------------------------------------------

\Sep % Extra whitespace after the end of a major section


\CVSection{Skills}{

%------------------------------------------------
I am skilled in the following languages and tools.
\begin{itemize}   
    \item Isabelle proof assistant, with which I developed a certified sequence solver.
    \item Coq proof assistant. I learned Coq when I was a PhD student, but unfortunately, I never used Coq in real projects.
    \item Event-B, which is a formal method for system-level modeling and analysis, with which I proof the code generation of SIGNAL language.
    \item \href{https://mlir.llvm.org/}{LLVM MLIR}, an infrastructure for building reusable and extensible compilers.  
    \item Git, with a team of more than 6 developers.
    \item Programming languages: OCaml C, C++, Python, Java, Rust, Solidity (smart contract language), OCaml.  
	\end{itemize}
	}
%----------------------------------------------------------------------------------------
%----------------------------------------------------------------------------------------




\Sep % Extra whitespace after the end of a major section

\CVSection{Publications}

%------------------------------------------------

\textbf{Conference Papers}
\begin{enumerate}
	\item \publication{chen2022solving} 
	 \item \publication{KanLRS22}
	 \item \publication{DBLP:conf/sp/JiaoK0S0020}
	\item \publication{DBLP:conf/issta/ChenYKQX19}
\end{enumerate}

\clearpage % Start a new page

\userinformation % Print your information in the left column

\framebreak % End of the first column

\begin{enumerate}
	\setcounter{enumi}{5}
	\item \publication{DBLP:conf/sigsoft/Kan14}
	\item \publication{DBLP:conf/icfem/KanHC16}
	\item \publication{DBLP:conf/memocode/BodeveixFK17}
	\item \publication{DBLP:conf/iceccs/JiangSZK019}
\end{enumerate}
~\\\\

\textbf{Journal Papers}

\begin{enumerate}
	\item \publication{yin2021safeosl}
	\item \publication{DBLP:journals/logcom/KanHCLH17}
	\item \publication{DBLP:journals/cj/KanH18}
	\item \publication{DBLP:journals/spe/KanH18}
	\item \publication{DBLP:journals/compsec/HuZLZKC21}
	\item \publication{DBLP:journals/access/LiKH17}
	\item \publication{DBLP:journals/corr/abs-1804-07608}
	\item \publication{yan2018location}
	\item \publication{kan2014bounded}(in Chinese)
\end{enumerate}

%------------------------------------------------

\Sep

%----------------------------------------------------------------------------------------

\CVSection{Projects}
\begin{enumerate}
	\item Research on Verification for the Memory Safety Mechanisms of Rust, Young Scientists Fund of the National Natural Science Foundation of China (PI) 250,000 CNY
	\item Securify: A compositional Approach of Building Security Verified System, \url{http://securify.sce.ntu.edu.sg/}(Participant)
	\item Research on Privacy Modeling and Verification in Evolving Cloud Computing, National Natural Science Foundation of China (Participant)
\end{enumerate}

\clearpage % Start a new page

\userinformation % Print your information in the left column

\framebreak % End of the first column

\CVSection{Teaching}


\CVItem{The Principles of Compiler Design}{Summer semester 2012, Nanjing University of Aeronautics and Astronautics, China}

\CVItem{Software Metrics}{Winter semester 2013, Nanjing University of Aeronautics and Astronautics, China}

\CVItem{Software Testing}{Summer semester 2018, Nanyang Technological University, Singapore}

\CVItem{An Introduction to Isabelle Interactive  Proof Assistant}{Department of Computer Science, Summer semester 2021, TU Kaiserslautern, Germany}

\CVItem{Logic and Verification Seminar}{Department of Computer Science, Summer semester 2021, TU Kaiserslautern, Germany}


\CVItem{An Introduction to Isabelle Interactive  Proof Assistant}{Department of Computer Science, Summer semester 2020, TU Kaiserslautern, Germany}

\CVItem{Logic and Verification Seminar}{Department of Computer Science, Summer semester 2020, TU Kaiserslautern, Germany}

\CVSection{Student Supervision}

\begin{enumerate}
	\item Xiaohua Yin (PhD student in Nanjing University of Aeronautics and Astronautics, China)
	\item Xinwen Hu (PhD student in Nanjing University of Aeronautics and Astronautics, China)
	\item Micha Schrader (Master student in TU Kaiserslautern, Germany)
\end{enumerate}

\Sep

\CVSection{Services}{ }

Assisting my supervisor in reviewing submissions from TACAS 2018, CCS 2018, FTSCS 2018, ATVA 2018.
Subreviewer of CSL 2022.

\Sep

\CVSection{Languages (Common European Framework of
	Reference for languages (CEFR))}{
%------------------------------------------------
\begin{itemize} 

    \item English C1.2 Level
    \item German A2 Level, and now learning B1.
    
  \end{itemize}
    	}
\Sep % Extra whitespace after the end of a major section
\end{document}
